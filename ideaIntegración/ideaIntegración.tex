\documentclass[reqno]{amsart}
\usepackage[spanish]{babel} % Para que algunas cosas estén en español
\usepackage[utf8]{inputenc} % Para hacer que LaTeX acepte tildes y eñes en el código
\usepackage[T1]{fontenc}
\usepackage{enumitem} % Para que el ambiente enumerate admita ciertos argumentos opcionales
\usepackage[symbol]{footmisc}
\usepackage{srcltx}

\newcommand{\dd}[1][y]{\if#1y\,\fi{\mathrm d}} % Differential
\newcommand{\norm}[2][y]{\if#1y\left\fi\lVert#2\if#1y\right\fi\rVert} % Norm
\newcommand{\abs}[2][y]{\if#1y\left\fi\lvert#2\if#1y\right\fi\rvert} % Absolute value

\newtheorem{thm}{Teorema}
\newtheorem{proposition}[thm]{Proposición}

\begin{document}
\title{Una idea para calcular ciertas integrales}
\maketitle

\section{Curvas geodésicas y teorema de la divergencia sobre la esfera}

Si $p, q \in \mathbb{S}^2 \subset \mathbb{R}^3$ no son idénticos ni antipodales, existe un único gran círculo $C(p,q)$ sobre $\mathbb{S}^2$ que los conecta.
La curva geodésica $\Gamma(p,q) \subset \mathbb{S}^2$ que conecta a $p$ y $q$ es la más corta de las dos curvas en las que $p$ y $q$ cortan a $C(p,q)$.

\begin{proposition}\label{pro:param} La curva $\Gamma(p,q)$ admite la siguiente parametrización:
%
\begin{equation*}
\Gamma(p,q) = \left\{ x(\zeta;p,q) = \cos(\zeta) \, p + \sen(\zeta) \, \frac{q - (p\cdot q)p}{\norm{q - (p\cdot q)p}} \mid \zeta \in [0, \arccos(p\cdot q)] \right\}.
\end{equation*}
%
\begin{proof} Ejercicio.
\end{proof}
\end{proposition}
%
Las siguientes fórmulas se obtienen inmediatamente:
%
\begin{equation}\label{param-aux}
\begin{gathered}
\abs{x(\zeta;p,q)} = 1, \quad
\frac{\mathrm{d}}{\mathrm{d}\zeta} x(\zeta;p,q) = -\sen(\zeta) \, p + \cos(\zeta) \, \frac{q - (p\cdot q)p}{\norm{q - (p\cdot q)p}},\\
\abs{\frac{\mathrm{d}}{\mathrm{d}\zeta} x(\zeta;p,q)} = 1, \quad
\frac{\mathrm{d}}{\mathrm{d}\zeta} x(\zeta;p,q) \cdot x(\zeta;p,q) = 0,\\
\frac{\mathrm{d}}{\mathrm{d}\zeta} x(\zeta;p,q) \times x(\zeta;p,q) = \frac{q \times p}{\norm{q \times p}}.
\end{gathered}
\end{equation}

Sean $p_0, p_1, \dotsc, p_{N-1}$ miembros de $\mathbb{S}^2$ tales que $p_i$ y $p_{i+1}$ no son idénticos ni antipodales para todo $i \in \{0, \dotsc, N-1\}$ (aquí y en lo que sigue usaremos la convención $p_N = p_0$) y sea $P$ una región encerrada por las curvas geodésicas $\Gamma(p_i, p_{i+1})$, $i \in \{0, \dotsc, N-1\}$, en sentido antihorario.
Sea $\sigma$ la medida de superficie sobre $\mathbb{S}^2$, $l$ la medida lineal sobre $\partial_{\mathbb{S}^2} P$, que a su vez es la frontera de $P$ respecto a la topología de subespacio que $\mathbb{S}^2$ hereda de $\mathbb{R}^3$, $F \colon \mathbb{S}^2 \to \mathbb{R}^3$ una función suficientemente regular y $\nu_P \colon \partial_{\mathbb{S}^2} P \to \mathbb{R}^3$ el vector unitario que en para $l$-casi-todo punto de su dominio es normal y exterior a $P$ y tangente a $\mathbb{S}^2$.
Entonces se cumple la siguiente variante del teorema de la divergencia:
%
\begin{equation}\label{manifoldDivergenceTheorem}
\int_P \operatorname{div}_{\mathbb{S}^2}(F(x)) \dd\sigma(x)
= \int_{\partial_{\mathbb{S}^2} P} F(x) \cdot \nu_P(x) \dd l(x)
\end{equation}
%

La integral del lado derecho de \eqref{manifoldDivergenceTheorem} puede escribirse como la suma de las integrales correspondientes sobre $\Gamma(p_i, p_{i+1})$, $i \in \{0, \dotsc, N-1\}$.
Cada una de estas integrales, a su vez, mediante la parametrización dada en la proposición \ref{pro:param}, puede escribirse como
%
\begin{equation*}
\int_0^{\arccos(p_i, p_{i+1})} F(x(\zeta;p_i,p_{i+1})) \cdot \nu_P(x(\zeta;p_i,p_{i+1})) \abs{\frac{\mathrm{d}}{\mathrm{d}\zeta} x(\zeta;p_i,p_{i+1})} \dd \zeta.
\end{equation*}
%
Usando las fórmulas en \eqref{param-aux} obtenemos que $\nu_P(x(\zeta;p_i,p_{i+1}))$ es el producto cruz entre $\frac{\mathrm{d}}{\mathrm{d}\zeta} x(\zeta;p_i,p_{i+1})$ y $x(\zeta;p_i,p_{i+1})$ (informalmente, derecha es el producto cruz entre adelante y arriba) y así el lado derecho de \eqref{manifoldDivergenceTheorem} toma la forma
%
\begin{equation*}
\sum_{i=0}^{N-1} \int_0^{\arccos(p_i,p_{i+1})} F(x(\zeta;p_i,p_{i+1})) \dd \zeta \cdot \frac{p_{i+1} \times p_i}{\norm{p_{i+1} \times p_i}}
\end{equation*}
%

Por otro lado, $\operatorname{div}_{\mathbb{S}^2}(F)$ es sencillamente la divergencia convencional de $\mathbb{R}^3$ aplicada a $\mathbb{R}^3 \ni x \mapsto F(x/\norm{x}) \in \mathbb{R}^3$ y restringida a $\mathbb{S}^2$.
Equivalentemente, si $\xi(\varphi,\theta) = \left[ \cos(\varphi) \sen(\theta), \sen(\varphi) \sen(\theta), \cos(\theta) \right]^T$,
%
\begin{multline}
\operatorname{div}_{\mathbb{S}^2} F(\xi(\varphi,\theta))\\
= 2 \, \underbrace{F(\xi(\varphi,\theta)) \cdot \hat r(\varphi,\theta)}_{\text{Componente radial}} + \frac{1}{\sen(\theta)} \frac{\partial}{\partial \theta}\left( \underbrace{F(\xi(\varphi,\theta)) \cdot \hat \theta(\varphi,\theta)}_{\text{Componente polar}} \sen(\theta) \right)\\ + \frac{1}{\sen(\theta)} \frac{\partial}{\partial \varphi}\left( \underbrace{F(\xi(\varphi,\theta)) \cdot \hat \varphi(\varphi,\theta)}_{\text{Componente azimutal}} \right),
\end{multline}
%
donde $\hat r$, $\hat \theta$ y $\hat \varphi$ son los campos vectoriales unitarios asociados al sistema de coordenadas esféricas expresados como funciones de $\varphi$ y $\theta$; esto es,
%
\begin{equation*}
\hat r(\varphi,\theta) = \begin{bmatrix} \cos(\varphi) \sen(\theta)\\ \sen(\varphi) \sen(\theta)\\ \cos(\theta) \end{bmatrix}, \quad
\hat \theta(\varphi,\theta) = \begin{bmatrix} \cos(\varphi) \cos(\theta)\\ \sen(\varphi) \cos(\theta)\\ -\sen(\theta) \end{bmatrix}, \quad
\hat \varphi(\varphi,\theta) = \begin{bmatrix} -\sen(\varphi)\\ \cos(\varphi)\\ 0 \end{bmatrix}.
\end{equation*}
%

\section{La idea}

Sea $T$ un triángulo de vértices $(\varphi_0, \theta_0)$, $(\varphi_1, \theta_1)$, $(\varphi_2, \theta_2)$ dados en sentido antihorario; esto es,
%
\begin{equation}\label{T}
T = \left\{ \sum_{i=0}^2 t_i (\varphi_i, \theta_i) \mid 0 \leq t_0, t_1, t_2 \leq 1 \ \wedge \ t_0+t_1+t_2=1 \right\}.
\end{equation}
%

Deseamos computar la integral\footnote{Omití el radio del planeta Tierra al cuadrado y la densidad de población de la comuna a la que $T$ está asociado para no sobrecargar la notación}
%
\begin{equation}\label{I1}
I_{T,x} := \int_T x(\varphi, \theta) \, \dd S(\varphi, \theta),
\end{equation}
%
donde $x(\varphi,\theta) = \cos(\varphi) \sen(\theta)$ y $\dd S(\varphi, \theta) = \sen(\theta) \dd(\varphi,\theta)$\footnote{Esto corresponde a la integral de la primera coordenada cartesiana sobre la región sobre la superficie de la esfera que sale de aplicar a $T$ la transformación $(\varphi, \theta) \mapsto (\cos(\varphi) \sen(\theta), \sen(\varphi) \sen(\theta), \cos(\theta))$ (notar que esto implica que la variable $\theta$ corresponde a una colatitud y que los segmentos de la frontera de $T$ no son mapeados a curvas geodésicas sobre la esfera).}.
Por lo tanto,
%
\begin{equation}\label{I2}
I_{T,x} := \int_T \cos(\varphi) \sen(\theta)^2 \dd(\varphi, \theta).
\end{equation}
%
Para evitar calcular integrales bidimensionales explícitamente, podemos aplicar el teorema de la divergencia a \eqref{I2} si conseguimos identificar una antidivergencia del integrando.
Importantemente, el operador divergencia que corresponde es el cartesiano\footnote{Informalmente uno diría que $T$ es lo que sale después de pasarle la aplanadora a la región correspondiente sobre la superficie de la esfera; matemáticamente el fondo del asunto es que en \eqref{I2} aparece la medida de Lebesgue convencional.} respecto a las coordenadas $(\varphi, \theta)$.
Jugando con Mathematica encontré que la función vectorial $F$ definida por
%
\begin{equation}\label{F}
F(\varphi, \theta) = \frac{1}{10}\begin{bmatrix}
\sen(\varphi) \left( 5 - \cos(2\theta) \right)\\[2ex]
-2 \cos(\varphi) \sen(2\theta)
\end{bmatrix}
\end{equation}
%
satisface
%
\begin{equation*}
\operatorname{div}(F)(\varphi,\theta)
= \frac{\partial F_1(\varphi, \theta)}{\partial \varphi} + \frac{\partial F_2(\varphi, \theta)}{\partial \theta}
= \cos(\varphi) \sen(\theta)^2.
\end{equation*}
%
Por lo tanto,
%
\begin{equation}\label{I3}
I_{T,x} = \int_{\partial T} F(\varphi, \theta) \cdot \nu_T(\varphi,\theta) \dd l(\varphi, \theta)
= \sum_{i=0}^2 \int_{E_i} F(\varphi, \theta) \cdot \nu_T(\varphi,\theta) \dd l(\varphi, \theta),
\end{equation}
%
donde $\nu_T$ es el vector unitario exterior normal a $T$ definido sobre $\partial T$, $l$ es la medida lineal sobre $\partial T$ y para cada $i \in \{0, 1, 2\}$, $E_i$ es el segmento que une a $(\varphi_i, \theta_i)$ con $(\varphi_{i+1}, \theta_{i+1})$, donde por supuesto los subíndices se interpretan módulo $3$; esto es,
%
\begin{equation}\label{Ei}
(\forall\,i\in\{0,1,2\}) \quad E_i = \left\{ (1-s) (\varphi_i, \theta_i) + s (\varphi_{i+1}, \theta_{i+1}) \mid 0 \leq s \leq 1 \right\}.
\end{equation}
%
Ahora, usando la parametrización de $E_i$ sugerida por \eqref{Ei}, las últimas integrales en \eqref{I3} se pueden escribir en la forma
%
\begin{multline}\label{lineIntegrals}
\int_{E_i} F(\varphi, \theta) \cdot \nu_T(\varphi,\theta) \dd l(\varphi, \theta)\\
= \int_0^1 F\big( (1-s) (\varphi_i, \theta_i) + s (\varphi_{i+1}, \theta_{i+1}) \big) \cdot \left[ \frac{1}{\abs{E_i}} \begin{bmatrix} 0 & 1\\-1 & 0\end{bmatrix} \begin{bmatrix} \varphi_{i+1}-\varphi_i \\ \theta_{i+1} - \theta_i \end{bmatrix} \right] \abs{E_i} \dd s\\
= \int_0^1 F\big( (1-s) (\varphi_i, \theta_i) + s (\varphi_{i+1}, \theta_{i+1}) \big) \cdot \begin{bmatrix} \theta_{i+1}-\theta_i\\ -(\varphi_{i+1} - \varphi_i) \end{bmatrix} \dd s.
\end{multline}
%
La última forma de la integral puede muy fácilmente ser alimentada a las excelentes rutinas de integración unidimensional que existen para Python, etc.

\section{Posibles refinamientos}

\begin{itemize}[leftmargin=*]
\item La última integral de \eqref{lineIntegrals} se puede calcular en forma analítica también.
Quizás efectuando la suma en \eqref{I3} de las expresiones resultantes se obtenga algo relativamente sencillo.
\item Para calcular las integrales
%
\begin{equation*}
I_{T,y} := \int_T y(\varphi, \theta) \, \dd S(\varphi, \theta)
\quad\text{y}\quad
I_{T,z} := \int_T z(\varphi, \theta) \, \dd S(\varphi, \theta),
\end{equation*}
%
donde $y(\varphi, \theta) = \sen(\varphi) \sen(\theta)$ y $z(\varphi, \theta) = \cos(\theta)$, probablemente no sea difícil hallar antidivergencias de los integrandos directamente o adaptando la antidivergencia $F$ dada en \eqref{F}.
\item Alternativamente uno podría tratar de expresar $I_{T,y}$ como $I_{Q(T),x}$, donde $Q$ es una rotación adecuada (similarmente para $I_{T,z}$).
\item \textbf{[Especulatorio]} ¿Qué hay si $T$, en vez de un triángulo, fuese la imagen por la transformación cartesiano-a-esférico de un verdadero triángulo esférico (uno donde los vértices son conectados por curvas geodésicas)?
\item \textbf{[Especulatorio]} ¿Habría alguna simplificación si usásemos teoremas de la divergencia especializados a regiones sobre superficies directamente sobre regiones sobre la esfera?
\end{itemize}

\end{document}
