\documentclass[reqno]{amsart}
\usepackage[spanish]{babel} % Para que algunas cosas estén en español
\usepackage[utf8]{inputenc} % Para hacer que LaTeX acepte tildes y eñes en el código
\usepackage[T1]{fontenc}
\usepackage{enumitem} % Para que el ambiente enumerate admita ciertos argumentos opcionales
\usepackage[symbol]{footmisc}
\usepackage{srcltx}

\newcommand{\dd}[1][y]{\if#1y\,\fi{\mathrm d}} % Differential
\newcommand{\norm}[2][y]{\if#1y\left\fi\lVert#2\if#1y\right\fi\rVert} % Norm
\newcommand{\abs}[2][y]{\if#1y\left\fi\lvert#2\if#1y\right\fi\rvert} % Absolute value
\newcommand{\psp}[1]{\sp{(#1)}} % Parenthesized superscript

\newtheorem{thm}{Teorema}
\newtheorem{proposition}[thm]{Proposición}

\begin{document}
\title{Una idea para calcular ciertas integrales}
\maketitle

\section{Curvas geodésicas y teorema de la divergencia sobre la esfera}

Si $p, q \in \mathbb{S}^2 \subset \mathbb{R}^3$ no son idénticos ni antipodales, existe un único gran círculo $C(p,q)$ sobre $\mathbb{S}^2$ que los conecta.
La curva geodésica $\Gamma(p,q) \subset \mathbb{S}^2$ que conecta a $p$ y $q$ es la más corta de las dos curvas en las que $p$ y $q$ cortan a $C(p,q)$.
Denotaremos el ángulo entre $p$ y$ q$ por $\angle(p,q) := \arccos(p\cdot q)$.

\begin{proposition}\label{pro:param} La curva $\Gamma(p,q)$ admite la siguiente parametrización:
%
\begin{equation*}
\Gamma(p,q) = \left\{ x(\zeta;p,q) = \cos(\zeta) \, p + \sen(\zeta) \, \frac{q - (p\cdot q)p}{\norm{q - (p\cdot q)p}} \mid \zeta \in [0, \angle(p,q)] \right\}.
\end{equation*}
%
\begin{proof} Ejercicio.
\end{proof}
\end{proposition}
%
Las siguientes fórmulas se obtienen inmediatamente:
%
\begin{equation}\label{param-aux}
\begin{gathered}
\abs{x(\zeta;p,q)} = 1, \quad
\frac{\mathrm{d}}{\mathrm{d}\zeta} x(\zeta;p,q) = -\sen(\zeta) \, p + \cos(\zeta) \, \frac{q - (p\cdot q)p}{\norm{q - (p\cdot q)p}},\\
\abs{\frac{\mathrm{d}}{\mathrm{d}\zeta} x(\zeta;p,q)} = 1, \quad
\frac{\mathrm{d}}{\mathrm{d}\zeta} x(\zeta;p,q) \cdot x(\zeta;p,q) = 0,\\
\frac{\mathrm{d}}{\mathrm{d}\zeta} x(\zeta;p,q) \times x(\zeta;p,q) = \frac{q \times p}{\norm{q \times p}}.
\end{gathered}
\end{equation}

Denotaremos por $\sigma$ la medida de superficie sobre $\mathbb{S}^2$.

\begin{proposition}\label{pro:sphereDiv}
Sea $P \subset \mathbb{S}^2$ una región sobre la esfera de frontera (en la topología de subespacio) $\partial_P$ y sea $l$ la medida de línea sobre $\partial P$.
Sea $\nu_P \colon \partial P \to \mathbb{R}^3$ una función a valores vectoriales que en $l$-casi todo punto de $\nu_P$ es normal y exterior a $P$ y tangente a $\mathbb{S}^2$.
Sea $F \colon \mathbb{S}^2 \to \mathbb{R}^3$ una función lo suficientemente regular.
Entonces,
%
\begin{equation}\label{sphereDiv}
\int_P \operatorname{div}_{\mathbb{S}^2}(F(x)) \dd\sigma(x)
= \int_{\partial P} F(x) \cdot \nu_P(x) \dd l(x),
\end{equation}
%
donde
%
\begin{subequations}\label{sphereDivOp}
\begin{equation}
(\forall\,x\in\mathbb{S}^2) \quad \operatorname{div}_{\mathbb{S}^2}(F(x)) = \left.\operatorname{div}(\hat F)\right|_{\mathbb{S}^2}(x) - 2 \, F(x) \cdot x,
\end{equation}
%
donde, a su vez, $\hat F \colon \mathbb{R}^3 \setminus \{0\} \to \mathbb{R}^3$ está definida por
%
\begin{equation}
\hat F(x) = F\left(\frac{x}{\norm{x}}\right).
\end{equation}
\end{subequations}
%
\begin{proof}[Esbozo de la demostración]
Aplicando el teorema de la divergencia convencional de $\mathbb{R}^3$ a la función $\hat F$ en la región $V = \{ x \in \mathbb{R}^3 \mid \norm{x}-1 \in (-\Delta R, \Delta R) \ \wedge\ x/\norm{x} \in P \}$, $\Delta R \in (0,1)$, obtenemos
%
\begin{multline*}
\int_V \operatorname{div}(\hat F(x)) \dd x = \int_{\{r y \mid y \in \partial P \ \wedge \ r-1 \in (-\Delta R,\Delta R)\}} F\left(\frac{x}{\norm{x}}\right) \cdot \nu_P\left(\frac{x}{\norm{x}}\right) \dd S(x)\\ + \int_{(1+\Delta R) P} F\left(\frac{x}{\norm{x}}\right) \cdot \frac{x}{\norm{x}} \dd S(x) - \int_{(1-\Delta R) P} F\left(\frac{x}{\norm{x}}\right) \cdot \frac{x}{\norm{x}} \dd S(x),
\end{multline*}
%
donde denotamos por $S$ a la medida de superficie genérica.
Dividiendo ambos lados por $2 \Delta R$ y tomando el límite lateral $\Delta R \to 0_+$, obtenemos el resultado deseado.
\end{proof}
\end{proposition}

El operador $\operatorname{div}_{\mathbb{S}^2}$ puede expresarse mediante los ángulos polar $\theta$ y azimutal $\varphi$ de las coordenadas esféricas.
Si $\xi(\theta,\varphi) = \left[ \cos(\varphi) \sen(\theta), \sen(\varphi) \sen(\theta), \cos(\theta) \right]^T$,
%
\begin{multline}
\operatorname{div}_{\mathbb{S}^2} F(\xi(\varphi,\theta))
= \frac{1}{\sen(\theta)} \frac{\partial}{\partial \theta}\left( \underbrace{F(\xi(\varphi,\theta)) \cdot \hat \theta(\varphi,\theta)}_{\text{Componente polar}} \sen(\theta) \right)\\ + \frac{1}{\sen(\theta)} \frac{\partial}{\partial \varphi}\left( \underbrace{F(\xi(\varphi,\theta)) \cdot \hat \varphi(\varphi,\theta)}_{\text{Componente azimutal}} \right),
\end{multline}
%
donde $\hat r$, $\hat \theta$ y $\hat \varphi$ son los campos vectoriales unitarios asociados al sistema de coordenadas esféricas expresados como funciones de $\theta$ y $\varphi$; esto es,
%
\begin{equation*}
\hat r(\theta,\varphi) = \begin{bmatrix} \cos(\varphi) \sen(\theta)\\ \sen(\varphi) \sen(\theta)\\ \cos(\theta) \end{bmatrix}, \quad
\hat \theta(\theta,\varphi) = \begin{bmatrix} \cos(\varphi) \cos(\theta)\\ \sen(\varphi) \cos(\theta)\\ -\sen(\theta) \end{bmatrix}, \quad
\hat \varphi(\theta,\varphi) = \begin{bmatrix} -\sen(\varphi)\\ \cos(\varphi)\\ 0 \end{bmatrix}.
\end{equation*}
%

Sean $p\psp{0}, p\psp{1}, \dotsc, p\psp{N-1}$ miembros de $\mathbb{S}^2$ tales que $p\psp{i}$ y $p\psp{i+1}$ no son idénticos ni antipodales para todo $i \in \{0, \dotsc, N-1\}$ (aquí y en lo que sigue usaremos la convención $p\psp{N} = p\psp{0}$) y sea $P$ una región encerrada por las curvas geodésicas $\Gamma(p\psp{i}, p\psp{i+1})$, $i \in \{0, \dotsc, N-1\}$, en sentido antihorario.

La integral del lado derecho de \eqref{sphereDiv} en la proposición \ref{pro:sphereDiv} puede en este caso escribirse como la suma de las integrales correspondientes sobre $\Gamma(p\psp{i}, p\psp{i+1})$, $i \in \{0, \dotsc, N-1\}$.
Cada una de estas integrales, a su vez, mediante la parametrización dada en la proposición \ref{pro:param}, puede escribirse como
%
\begin{equation*}
\int_0^{\angle(p\psp{i}, p\psp{i+1})} F(x(\zeta;p\psp{i},p\psp{i+1})) \cdot \nu_P(x(\zeta;p\psp{i},p\psp{i+1})) \abs{\frac{\mathrm{d}}{\mathrm{d}\zeta} x(\zeta;p\psp{i},p\psp{i+1})} \dd \zeta.
\end{equation*}
%
Usando las fórmulas en \eqref{param-aux} obtenemos que $\nu_P(x(\zeta;p\psp{i},p\psp{i+1}))$ es el producto cruz entre $\frac{\mathrm{d}}{\mathrm{d}\zeta} x(\zeta;p\psp{i},p\psp{i+1})$ y $x(\zeta;p\psp{i},p\psp{i+1})$ (informalmente, derecha es el producto cruz entre adelante y arriba) y así \eqref{sphereDiv} se particulariza a
%
\begin{equation}\label{lineIntegralDecomposition}
\int_P \operatorname{div}_{\mathbb{S}^2} F(x)) \dd\sigma(x)
= \sum_{i=0}^{N-1} \int_0^{\angle(p\psp{i},p\psp{i+1})} F(x(\zeta;p\psp{i},p\psp{i+1})) \dd \zeta \cdot \frac{p\psp{i+1} \times p\psp{i}}{\norm{p\psp{i+1} \times p\psp{i}}}.
\end{equation}
%


\subsection{Aplicación a integral de función uno}

La función $F \colon \mathbb{S}^2 \to \mathbb{R}^3$ definida por $(\forall\,x\in\mathbb{S}^2)\ F(x) = [0, 0, x_3/(1-x_3^2)]^T$ satisface $\operatorname{div}_{\mathbb{S}^2} F = 1$.
En este caso (abreviamos $\hat q = \frac{q - (p\cdot q)p}{\norm{q - (p\cdot q)p}}$),
%
\begin{multline}\label{aux-1}
\int_0^{\angle(p,q)} F(x(\zeta;p,q))_3 \dd \zeta
= \int_0^{\angle(p,q)} \frac{x(\zeta;p,q)_3}{1-x(\zeta;p,q)_3^2} \dd\zeta\\
= \int_0^{\angle(p,q)} \frac{\cos(\zeta) \, p_3 + \sen(\zeta) \, \hat q_3}{1 - \left(\cos(\zeta) \, p_3 + \sen(\zeta) \, \hat q_3\right)^2} \dd \zeta\\
= \left. \frac{1}{\sqrt{1 - p_3^2 - \hat q_3^2}} \arctan\left( \frac{\sen(\zeta) \, p_3 - \cos(\zeta) \hat q_3}{\sqrt{1 - p_3^2 - \hat q_3^2}} \right) \right|_{\zeta=0}^{\zeta=\angle(p,q)}.
\end{multline}
%
Esta forma ya es útil para explotar \eqref{lineIntegralDecomposition}.
Sin embargo, es mi sospecha que se puede simplificar aún más.

\subsection{Aplicación a integral de función primera coordenada cartesiana}
La función $F \colon \mathbb{S}^2 \to \mathbb{R}^3$ definida por $(\forall\,x \in\mathbb{S}^2)\ F(x) = [-1/2,0,0]^T$ (¡constante!) satisface $\operatorname{div}_{\mathbb{S}^2}(F)(x) = x_1$.
En este caso,
%
\begin{equation*}
\int_0^{\angle(p,q)} F(x(\zeta;p,q)) \dd \zeta
=\int_0^{\angle(p,q)} \begin{bmatrix} -1/2 \\ 0 \\ 0 \end{bmatrix} \dd \zeta
= \begin{bmatrix} -\angle(p,q)/2 \\ 0 \\ 0 \end{bmatrix}.
\end{equation*}
%
Por lo tanto, por \eqref{lineIntegralDecomposition}
%
\begin{equation*}
\begin{split}
\int_P x_1 \dd \sigma(x) & = -\frac{1}{2} \sum_{i=0}^{N-1} \angle(p\psp{i},p\psp{i+1}) \frac{(p\psp{i+1} \times p\psp{i})_1}{\norm{p\psp{i+1} \times p\psp{i}}}\\
& = \frac{1}{2} \sum_{i=0}^{N-1} \frac{\angle(p\psp{i},p\psp{i+1})}{\sqrt{1-(p\psp{i} \cdot p\psp{i+1})^2}} (p\psp{i+1}_3 p\psp{i}_2 - p\psp{i+1}_2 p\psp{i}_3).
\end{split}
\end{equation*}

\subsection{Otras coordenadas cartesianas}

Como la imagen a través de una rotación de una curva geodésica también es una curva geodésica, una rutina que calcule la integral de la función $\mathbb{S}^3 \ni x \mapsto x_1 \in \mathbb{R}$ en un polígono definido por curvas geodésicas puede ser combinada con rotaciones adecuadas para inmediatamente producir las rutinas correspondientes a las funciones $\mathbb{S}^3 \ni x \mapsto x_2 \in \mathbb{R}$ y $\mathbb{S}^3 \ni x \mapsto x_3 \in \mathbb{R}$.

\end{document}
